\documentclass[english,11pt,a4paper]{article}

\usepackage[english]{babel}
\usepackage{graphicx}
\graphicspath{{./}{../}{images/}{../images/}}

\usepackage{amsmath,amssymb,amsthm,bm,gensymb}
\usepackage{mathrsfs}
\usepackage{mathtools}

% ---Simons shizzle: afblijven!---
\usepackage[dvipsnames]{xcolor} % maakt het mogelijk zelf kleuren te definiëren
\definecolor{darkPurple}{RGB}{80,0,80} % definieer donkerpaars
\definecolor{lightGray}{RGB}{150, 150, 150}
\usepackage{hyperref} % maakt links en cross-references interactief
\usepackage[all]{hypcap} % zorgt dat hyperref beter naar captions kan refereren
\hypersetup{colorlinks=true,urlcolor=blue,citecolor=darkPurple,linkcolor=darkPurple} % bepaalt kleuren van de links
\newcommand{\cref}[2]{\hyperref[#2]{#1~\ref*{#2}}} % voeg woord aan link toe

\usepackage[labelfont=bf,margin=0.5cm]{caption} % interface voor captions
\usepackage{subcaption}

\usepackage{microtype} % mierenneukerij
%%% Fix bug in microtype (https://tex.stackexchange.com/q/619410/149123)
\makeatletter
\def\MT@is@opt@char#1\iffontchar#2\char#3\else#4\fi\relax{%
  \MT@ifempty{#1}{%
    \iffontchar#2%
      \expandafter\chardef
        \csname\MT@encoding\MT@detokenize@c\@tempa\endcsname=#3\relax
    \fi
  }\relax
}
\makeatother
%%% Einde bug-fix
\usepackage{xfrac} % mooie breuken

\usepackage[utf8]{inputenc} % maakt onder andere accenten mogelijk (¨´` ...)
\usepackage[T1]{fontenc} % zorgt dat utf8-characters goed getoond worden
\usepackage{lmodern} % maakt type-setting compatibel met utf8

\usepackage{parskip} % voegt enters tussen paragrafen toe

\usepackage{siunitx}
% geef fysische grootheden op een consistente manier weer
\sisetup{retain-explicit-plus, list-final-separator={\,en\,}, list-pair-separator={\,en\,}, range-phrase={\,tot\,}, separate-uncertainty, per-mode=fraction, table-align-uncertainty=true}
\usepackage{textcomp} % vermijd conflicten tussen microtype en siunitx

\usepackage[backend=biber]{biblatex}
\usepackage{url}
\setcounter{biburllcpenalty}{7000}
\setcounter{biburlucpenalty}{8000}
\usepackage{csquotes}
\usepackage{float}
\addbibresource{biblio.bib}
% ---Einde Simons shizzle---

\usepackage{geometry}
\geometry{
 a4paper,
 vmargin={2.3cm, 3cm}
}

\usepackage{import}

%%%%%%% AUXILIARY FUNCTIONS %%%%%%%%
\usepackage{amsmath}
\DeclareMathOperator{\ET}{ET} % entity table
\DeclareMathOperator{\FT}{FT} % function table
\DeclareMathOperator{\attrs}{attrs} % attributes
\DeclareMathOperator{\allattrs}{allattrs} % attributes, including from its ancestors
\DeclareMathOperator{\inp}{inputs} % inputs
\DeclareMathOperator{\outp}{output} % output
\DeclareMathOperator{\op}{op} % operation
\DeclareMathOperator{\fexpr}{expr} % function expression
\DeclareMathOperator{\pexpr}{expr*} % path segment expression
\DeclareMathOperator{\ancestors}{ancestors} % set of ancestors
\DeclareMathOperator{\comparable}{comparable} % check whether two basic types are comparable
\DeclareMathOperator{\overlap}{overlap} % check whether cardinalities overlap
\DeclareMathOperator{\subcardinality}{subcardinality} % check whether a cardinality range is enclosed by another cardinality range
\DeclareMathOperator{\union}{union} % union of cardinalities
\DeclareMathOperator{\join}{join} % join of two types
\DeclareMathOperator{\maybeempty}{maybeempty} % all attributes may be empty

\newcommand{\lv}[1]{#1^{*}} % list version of an operator


%%%%%%%% CODE LISTINGS %%%%%%%%%%
\usepackage[final]{listings}

\usepackage[pdftex,dvipsnames]{xcolor}
\definecolor{codegreen}{rgb}{0,0.6,0}
\definecolor{codegray}{rgb}{0.5,0.5,0.5}
\definecolor{codepurple}{rgb}{0.58,0,0.82}
\definecolor{backcolour}{rgb}{0.95,0.95,0.92}
\lstdefinestyle{mystyle}{
    backgroundcolor=\color{backcolour},   
    commentstyle=\color{codegreen},
    keywordstyle=\color{magenta},
    numberstyle=\tiny\color{codegray},
    stringstyle=\color{codepurple},
    basicstyle=\ttfamily\small,
    breakatwhitespace=false,         
    breaklines=true,                 
    captionpos=b,     
    keepspaces=true,                 
    numbers=left,                    
    numbersep=5pt,                  
    showspaces=false,                
    showstringspaces=false,
    showtabs=false,                  
    tabsize=2
}

% Add background color to inline listings
% See https://tex.stackexchange.com/a/357339/149123
% and https://tex.stackexchange.com/a/408880/149123
%\usepackage{xpatch}
%\usepackage{realboxes}
%\makeatletter
%\xpretocmd\lstinline
%  {%
%   \bgroup\fboxsep=1.5pt
%   \Colorbox{backcolour}\bgroup\kern-\fboxsep\vphantom{\ttfamily\char`\\y}%
%   \appto\lst@DeInit{\kern-\fboxsep\egroup\egroup}%
%  }{}{}
%\makeatother

% Patch for lstinline inside math mode (see https://tex.stackexchange.com/a/127018/149123)
\usepackage{letltxmacro}
\newcommand*{\SavedLstInline}{}
\LetLtxMacro\SavedLstInline\lstinline
\DeclareRobustCommand*{\lstinline}{%
  \ifmmode
    \let\SavedBGroup\bgroup
    \def\bgroup{%
      \let\bgroup\SavedBGroup
      \hbox\bgroup
    }%
  \fi
  \SavedLstInline
}

\definecolor{eminence}{RGB}{108,48,130}
\lstdefinelanguage{Rosetta}{%
alsoletter={-},
keywords={-},
otherkeywords={% Operators
+, *, /, ->, =, <>
},
morekeywords=[2]{boolean,int,number}, % types go here
morekeywords=[3]{%
type,extends,func,enum,inputs,output,assign-output,not,or,and,only,single,multiple,exists,is,absent,contains,%
disjoint,all,any,count,only-element,if,then,else,True,False,empty,%
any,boolean,string,int,number,nothing,date,dateTime,zonedDateTime,time,%
true,false % keywords go here
},%
basicstyle={\sffamily},
keywordstyle=[2]{\itshape}, % style for types
keywordstyle=[3]{\ttfamily\color{eminence}}, %style for keywords
keepspaces,
mathescape % optional
}[keywords,comments,strings]%

\lstset{style=mystyle, language=Rosetta}

%%%%%%%%% SYNTAX DEFINITIONS %%%%%%%%%%%
\usepackage[nounderscore]{syntax}
\setlength{\grammarindent}{4.5em}

\renewcommand{\grammarlabel}[2]{\synt{#1}\hfill #2}

\renewcommand{\syntleft}{$\langle$\normalfont\scshape}
\renewcommand{\syntright}{$\rangle$}

\renewcommand{\litleft}{\bgroup\normalfont\ttfamily\frenchspacing}
\renewcommand{\litright}{\egroup}

\renewcommand{\ulitleft}{\normalfont\itshape}
\renewcommand{\ulitright}{}

\newcommand{\explain}[1]{\hfill\textit{#1}\\}
\newcommand{\ind}{\phantom{.}\hspace{1em}}

%%%%%%%%% MATH LIGATURES %%%%%%%%%%
\usepackage[ligature,reserved,shorthand]{semantic}
\usepackage{stmaryrd}
\mathlig{|[}{\left\llbracket}
\mathlig{|]}{\right\rrbracket}
\reservestyle[\lit]{\rosetta}{\mathinner}
\rosetta{type,extends,func,enum,inputs,output,assign-output,not,or,and,only,single,multiple,exists,is,absent,contains,%
disjoint,all,any,count,only-element,if,then,else,True,False,empty,%
any,boolean,string,int,number,nothing,date,dateTime,zonedDateTime,time,%
true,false,eq[=],neq[<>],+,-,*,/,proj[->]}

\newcommand{\emptylist}{[\,]}

%%%%%%%% INFERENCE RULES %%%%%%%%%
\usepackage{mathpartir}
\let\oldrule\rule
\renewcommand{\rule}[1]{\textnormal{\textsc{#1}}}

\newcommand{\derivation}[1]{\mathcal{#1}}
\newcommand{\OK}{\mathinner{\text{OK}}}

\newcommand{\ruledef}[3]{\inferrule*[vcenter,narrower=0.8]{#2}{#3} & \quad \textsc{\small #1}}
\newcommand{\axiomdef}[2]{#2 & \quad \textsc{\small #1}}

\newcommand{\emphrule}[1]{{\color{red}#1}}

\newcommand{\evaldef}[3]{\sexpr|[#2|]S &= #3 && \textsc{\small #1}}
% Definition for \letin; see https://tex.stackexchange.com/a/627476/149123
\ExplSyntaxOn
\NewDocumentCommand{\letin}{mm}
 {
  \internal_letin:nn { #1 } { #2 }
 }
\seq_new:N \l__internal_letin_assign_seq
\cs_new_protected:Nn \internal_letin:nn
 {
  \seq_set_split:Nnn \l__internal_letin_assign_seq { & } { #1 }
  \openup-\jot
  \begin{aligned}[t]
  &\textbf{let~}
  \begin{aligned}[t]
  &\seq_use:Nn \l__internal_letin_assign_seq {, \\ &} \\
  \end{aligned}\\
  &\textbf{in~} #2
  \end{aligned}
 }
\ExplSyntaxOff

%%%%%%%%% SEMANTIC DEFINITION %%%%%%%%%%%
\usepackage{amssymb}
\let\oldvec\vec
\renewcommand{\vec}[1]{\boldsymbol{\mathbf{#1}}} % vector
\DeclareMathOperator{\dom}{Dom}

\newcommand{\stype}{\mathcal{T}}
\newcommand{\sexpr}{\evalsymbol}
\newcommand{\sentity}{\mathcal{D}}
\newcommand{\sprog}{\mathcal{P}}

\newcommand{\set}[1]{\left\{\,#1\,\right\}}
\newcommand{\sboolean}{\mathbb{B}}
\newcommand{\strue}{\mathit{true}}
\newcommand{\sfalse}{\mathit{false}}
\newcommand{\sint}{\mathbb{Z}}
\newcommand{\snumber}{\mathbb{R}}
\newcommand{\sprim}{\text{Prim}}
\newcommand{\sent}{\text{Ent}}
\newcommand{\sdom}{\mathbb{D}}

\DeclareMathOperator{\sflatten}{flatten}
\DeclareMathOperator{\scount}{count}
\DeclareMathOperator{\seq}{equals}
\newcommand{\splus}[1]{\mathbin{\mathrm{plus}}_{#1}}
\newcommand{\ssubt}[1]{\mathbin{\mathrm{subtract}}_{#1}}
\newcommand{\smult}[1]{\mathbin{\mathrm{multiply}}_{#1}}
\newcommand{\sdiv}{\mathbin{\mathrm{divide}}}
\DeclareMathOperator{\sproject}{project}


%%%%%%%%%% DEFINITION BOXES %%%%%%%%%
\usepackage[many]{tcolorbox}
\usepackage{multicol}
% Enable single column multicols
\let\multicolmulticols\multicols
\let\endmulticolmulticols\endmulticols
\RenewDocumentEnvironment{multicols}{mO{}}
 {%
  \ifnum#1=1
    #2%
  \else % More than 1 column
    \multicolmulticols{#1}[#2]
  \fi
 }
 {%
  \ifnum#1=1
  \else % More than 1 column
    \endmulticolmulticols
  \fi
 }

\newcommand{\defboxisbreakable}{true}
\AddToHook{env/spec/begin}{\renewcommand{\defboxisbreakable}{false}}
\AddToHook{env/spec/end}{\renewcommand{\defboxisbreakable}{true}}

\usepackage{xargs}
\setlength{\columnseprule}{1pt}
\newenvironmentx{defbox}[2][1=1, 2=]{%
\begin{tcolorbox}[
enhanced,
breakable=\defboxisbreakable,
sharp corners=all,
colback=black!4!white,
toprule=1.5pt,bottomrule=1.5pt,
leftrule=0pt,rightrule=0pt,
left=1pt,right=1pt,
width=0.85\paperwidth,
center,
#2
]
\begin{multicols}{#1}
\begingroup
\allowdisplaybreaks
\addtolength{\jot}{0.5em}
}
{%
\endgroup
\end{multicols}
\end{tcolorbox} 
}

\usepackage{float,caption}
\floatstyle{plaintop}
\newfloat{spec}{tbp}{specs}
\floatname{spec}{Specification}
\captionsetup[spec]{labelsep=colon}
\ifundef\chapter{}{\numberwithin{spec}{chapter}}


\usepackage{afterpage, environ}
\long\def\breakablespecbody#1[#2][#3]{%
%\afterpage%
}
\NewEnviron{breakablespec}[2]{
\expandafter\breakablespecbody\expandafter{\BODY}[#1][#2]%
}



%%%%%%%%%% CODE GENERATION %%%%%%%%%%
\newcommand{\java}{\mathcal{J}}
\DeclareMathOperator{\javaid}{javaId}
\DeclareMathOperator{\freeid}{freeId}




\title{Specification of the Nouga DSL}
\author{Simon Cockx}
\date{\today}

\begin{document}

\maketitle



The goal of this work is two-fold. On the one hand it aims to eliminate flaws from the Rosetta language by formalizing its grammar and typing system. On the other hand it seeks to give solid ground for developers of code generators, giving a single source of truth about the intended semantics of generated code. Note that these goals also constitute two different audiences; one the developers of Rosetta, the other parties interested in translating Rosetta to a new language.

Throughout this document, $T$ and $S$ represent basic types and $C$ represents a cardinality.

\section{Syntax}

Metavariables: \lit*{D} and \lit*{E} range over entity names, \lit*{F} ranges over function names, \lit*{a} and \lit*{b} range over attribute and parameter names, \lit*{i} and \lit*{j} range over signed integers, \lit*{k} and \lit*{l} range over positive integers, \lit*{r} ranges over signed decimals. Whitespace is ignored.

\begin{defbox}[2]
\subimport{}{syntax-def}
\end{defbox}

\paragraph{Operator precedence} (note: this differs from Rosetta. The precedence of operators common with the C language are based on \url{https://en.cppreference.com/w/c/language/operator_precedence}.)
\begin{enumerate}
\item \lit{->} (projection), \lit{->} $a$ \lit{only exists}
\item \lit{only-element}
\item \lit{exists}, \lit{is absent}, \lit{count}
\item \lit{not}
\item \lit{*} (multiplication), \lit{/} (division)
\item \lit{+} (addition), \lit{-} (subtraction)
\item \lit{=} (equality), \lit{<>} (inequality)
\item \lit{contains}, \lit{disjoint}
\item \lit{and}
\item \lit{or}
\end{enumerate}

\paragraph{Syntactic sugar}
\begin{align*}
\<if> e_1 \<then> e_2 &\equiv \<if> e_1 \<then> e_2 \<else> \<empty> \\
\<empty> &\equiv [] \\
e \<is> \<absent> &\equiv \<not> (e \<exists>)
\end{align*}


Note: Nouga has a couple of differences compared to Rosetta.
\begin{enumerate}
\item Nouga replaces the multiple \lit{assign-output} statements with a single statement that fully defines the output of a function. Instead of defining one attribute of the output per \lit{assign-output} statement, you can use a record-like syntax to explicitely create an instance. (see the last option of expressions \synt{E})
\item Empty list literals are allowed.
\item In Nouga you can write \lit{not} expressions.
\item The \lit{only-element} keyword can be written behind any expression.
\item The \lit{only} \lit{exists} is restricted to expressions that end with a projection \lit{->} \lit*{a}. This simplifies the runtime model (i.e. code generators) as attributes in Nouga do not need to keep track of their parent.
\end{enumerate}


\section{Auxiliary definitions}

Entity table $\ET(D)$ is a mapping from data type names to data declarations. Function table $\FT(F)$ is a mapping from function names to function declarations.
\begin{defbox}
\subimport{}{auxiliary-def}
\end{defbox}


\section{Semantics}

Semantic domain: $\sdom$.

Single value: $\ssingledom$

\subsection{Semantics of Types}
\begin{defbox}[2]
\subimport{}{semantics-types-def}
\end{defbox}


\subsection{Semantical Algebra}
\begin{defbox}
\paragraph{Semantic algebra.}
\begin{align*}
A^0 &= Unit = \set{()} \\
A^n &= A \times A^{n-1} \\
A^{m:n} &= \bigcup_{k \in m..n} A^k \\
A^{*} &= A^{0:\infty}
\end{align*}

Note: from the above definition, $A^1$ formally equals $A \times Unit$, so elements of this set are of the form $(a, ())$ where $a \in A$. I might sometimes write $a$ instead of $(a, ())$ if it is clear from the context what is meant. (similar for $A^n$, where I will leave out the last element of the cartesian product)

\subimport{}{semantics-algebra-def}

Note: equality is checked deeply, i.e. recursively on attributes of records.


Given $f : A_1 \times \dots \times A_n \to B$ where $A_1, \dots, A_n \subset \sdom$ and $B \subset \sdom$, let
\begin{equation*}
\hat{f} : \sdom_\bot^n \to \sdom_\bot : \hat{f}(a_1, \dots, a_n) =
\begin{cases}
f(a_1, \dots, a_n), & (a_1, \dots, a_n) \in \dom{f} \\
\bot, & \text{otherwise.}
\end{cases}
\end{equation*}
\end{defbox}


\subsection{Semantics of Expressions}
\begin{defbox}
Some denotations depend on the type derivation of an expression. For this reason, I will evaluate typing derivations instead of expressions. However, because I only need this in a few cases, I will often omit the derivation, i.e. I will write $|[e|]$ instead of $|[\derivation{D} :: \emptyset |- e : T\ C|]$ if the type $T\ C$ and the derivation $\derivation{D}$ are unimportant.

Values $|[v|]$.
\begin{align*}
\axiomdef{E-True}{|[True|] = (true, ())}
\\
\axiomdef{E-False}{|[False|] = (false, ())}
\\
\axiomdef{E-Int}{|[i|] = (i, ())}
\\
\axiomdef{E-Number}{|[r|] = (r, ())}
\end{align*}

Expressions $|[e|]$.
\begin{align*}
\axiomdef{E-List}{|[\derivation{D} :: \emptyset |- [e_1, \dots, e_n] : T\ C|] = \sflatten{|[T|]}{n}{|[e_1|], \dots, |[e_n|]}}
\\
\axiomdef{E-Or}{|[e_1 \<or> e_2|] = \shor{|[e_1|]}{|[e_2|]}}
\\
\axiomdef{E-And}{|[e_1 \<and> e_2|] = \shand{|[e_1|]}{|[e_2|]}}
\\
\axiomdef{E-Not}{|[\<not> e|] = \shnot{|[e|]}}
\\
\axiomdef{E-Exists}{|[e \<exists>|] = 
\begin{cases}
(false, ()), & |[e|] = () \\
(true, ()), & \text{otherwise}
\end{cases}}
\\
\axiomdef{E-SingleExists}{|[e \<single> \<exists>|] = 
\begin{cases}
(true, ()), & \shcount{|[e|]} = (1, ()) \\
(false, ()), & \text{otherwise}
\end{cases}}
\\
\axiomdef{E-MultipleExists}{|[e \<multiple> \<exists>|] = 
\begin{cases}
(true, ()), & \shcount{|[e|]} = (k, ()) \land k \geq 2 \\
(false, ()), & \text{otherwise}
\end{cases}}
\\
\axiomdef{E-Contains}{|[e_1 \<contains> e_2|] = \shcontains{|[e_1|]}{|[e_2|]}}
\\
\axiomdef{E-Disjoint}{|[e_1 \<disjoint> e_2|] = \shdisjoint{|[e_1|]}{|[e_2|]}}
\\
\axiomdef{E-Equals}{|[e_1 \<eq> e_2|] = \sheq{|[e_1|]}{|[e_2|]}}
\\
\axiomdef{E-NotEquals}{|[e_1 \<neq> e_2|] = \shneq{|[e_1|]}{|[e_2|]}}
\\
\axiomdef{E-AllEquals}{|[e_1 \<all> \<eq> e_2|] = \shalleq{|[e_1|]}{|[e_2|]}}
\\
\axiomdef{E-AllNotEquals}{|[e_1 \<all> \<neq> e_2|] = \shallneq{|[e_1|]}{|[e_2|]}}
\\
\axiomdef{E-AnyEquals}{|[e_1 \<any> \<eq> e_2|] = \shanyeq{|[e_1|]}{|[e_2|]}}
\\
\axiomdef{E-AnyNotEquals}{|[e_1 \<any> \<neq> e_2|] = \shanyneq{|[e_1|]}{|[e_2|]}}
\\
\axiomdef{E-Plus}{|[\inferrule{\derivation{D}_1 :: \emptyset |- e_1 : T\ (1..1) \\ \derivation{D}_2 :: \emptyset |- e_2 : T\ (1..1)}{\emptyset |- e_1 \<+> e_2 : T\ (1..1)}|] = \shplus{|[T|]}{|[e_1|]}{|[e_2|]}}
\\
\axiomdef{E-Subt}{|[\inferrule{\derivation{D}_1 :: \emptyset |- e_1 : T\ (1..1) \\ \derivation{D}_2 :: \emptyset |- e_2 : T\ (1..1)}{\emptyset |- e_1 \<-> e_2 : T\ (1..1)}|] = \shsubt{|[T|]}{|[e_1|]}{|[e_2|]}}
\\
\axiomdef{E-Mult}{|[\inferrule{\derivation{D}_1 :: \emptyset |- e_1 : T\ (1..1) \\ \derivation{D}_2 :: \emptyset |- e_2 : T\ (1..1)}{\emptyset |- e_1 \<*> e_2 : T\ (1..1)}|] = \shmult{|[T|]}{|[e_1|]}{|[e_2|]}}
\\
\axiomdef{E-Div}{|[e_1 \</> e_2|] = \shdiv{|[e_1|]}{|[e_2|]}}
\\
\axiomdef{E-Count}{|[e \<count>|] = \shcount{|[e|]}}
\\
\axiomdef{E-Project}{|[\inferrule{\derivation{D} :: \emptyset |- e : D\ C}{\emptyset |- e \<proj> a : T\ C_a}|] = \shproject{|[D|]}{a}{|[e|]}}
\\
\axiomdef{E-If}{|[\<if> e_1 \<then> e_2 \<else> e_3|] = \shchoice{|[e_1|]}{|[e_2|]}{|[e_3|]}}
\\
\ruledef{E-Func}{\inp(F) = a_1\ T_1\ C_1, \dots, a_n\ T_n\ C_n}{|[F(e_1, \dots, e_n)|] = |[\;\left[\; a_1 \mapsto e_1, \dots, a_n \mapsto e_n \;\right]\op(F)|]}
\\
\ruledef{E-Construct}{\attrs(D) = a_1\ T_1\ C_1, \dots, a_n\ T_n\ C_n}{|[D \set{a_1: e_1, \dots, a_n: e_n}|] = (\set{a_1 = |[e_1|], \dots, a_n = |[e_n|]}, ())}
\\
\axiomdef{E-OnlyExists}{|[\inferrule{\emptyset |- e : D\ (1..1)}{\emptyset |- e \<proj> a \<only> \<exists> : \<boolean>\ (1..1)}|] = \shonlyexists{|[D|]}{a}{|[e|]}}
\\
\axiomdef{E-OnlyElement}{|[e \<only-element>|] = \shonlyelement{|[e|]}}
\end{align*}
\end{defbox}


Note: equality between two empty lists (i.e. true) is different than the usual equality with null (i.e. always false) in other programming languages (and the official Rosetta documentation).


\section{Typing}

\subsection{Declarative Typing}

Some auxiliary definitions.
\begin{align*}
\inf((l..u)) &= l \\
\sup((l..u)) &= u \\
\subcardinality((l_1..u_1), (l_2..u_2)) &= l_1 \geq l_2 \land u_1 \leq u_2 \\
\comparable(T_1, T_2) &= T_1 <: T_2 \lor T_2 <: T_1 \\
\overlap((l_1..u_1), (l_2..u_2)) &= u_1 \geq l_2 \land u_2 \geq l_1 \\
\lv{\comparable}(T_1\ C_1, T_2\ C_2) &= \comparable(T_1, T_2) \land \overlap(C_1, C_2) \\
\union((l_1..u_1), (l_2..u_2)) &= (\min(l_1, l_2)..\max(u_1, u_2))
\end{align*}

\begin{defbox}
Subtyping $S <: T$.
\begin{align*}
\axiomdef{S-Refl}{T <: T}
\\
\ruledef{S-Trans}{S <: U \\ U <: T}{S <: T}
\\
\axiomdef{S-Num}{\<int> <: \<number>}
\\
\ruledef{S-Extends}{\DT(D) = \<type> D \<extends> E: \dots}{D <: E}
\end{align*}

List subtyping $T_1\ C_1 \lv{<:} T_2\ C_2$.
\begin{align*}
\ruledef{S-Card}{T_1 <: T_2 \\ \subcardinality(C_1, C_2)}{T_1\ C_1 \lv{<:} T_2\ C_2}
\end{align*}

Typing rules $\Gamma |- e : T\ C$.
\begin{align*}
\ruledef{T-Or}{\Gamma |- e_1 : \<boolean>\ (1..1) \\ \Gamma |- e_2 : \<boolean>\ (1..1)}{\Gamma |- e_1 \<or> e_2 : \<boolean>\ (1..1)}
\\
\ruledef{T-And}{\Gamma |- e_1 : \<boolean>\ (1..1) \\ \Gamma |- e_2 : \<boolean>\ (1..1)}{\Gamma |- e_1 \<and> e_2 : \<boolean>\ (1..1)}
\\
\ruledef{T-Not}{\Gamma |- e : \<boolean>\ (1..1)}{\Gamma |- \<not> e : \<boolean>\ (1..1)}
\\
\ruledef{T-Exists}{\Gamma |- e : T\ C \\ \subcardinality((0..1), C)}{\Gamma |- e \<exists> : \<boolean>\ (1..1)}
\\
\ruledef{T-SingleExists}{\Gamma |- e : T\ C \\ \subcardinality((1..1), C) \\ C \neq (1..1)}{\Gamma |- e \<single> \<exists> : \<boolean>\ (1..1)}
\\
\ruledef{T-MultipleExists}{\Gamma |- e : T\ C \\ \subcardinality((1..2), C)}{\Gamma |- e \<multiple> \<exists> : \<boolean>\ (1..1)}
\\
\ruledef{T-Contains}{\Gamma |- e_1 : T_1\ C_1 \\ \Gamma |- e_2 : T_2\ C_2 \\ \comparable(T_1, T_2)}{\Gamma |- e_1 \<contains> e_2 : \<boolean>\ (1..1)}
\\
\ruledef{T-Disjoint}{\Gamma |- e_1 : T_1\ C_1 \\ \Gamma |- e_2 : T_2\ C_2 \\ \comparable(T_1, T_2)}{\Gamma |- e_1 \<disjoint> e_2 : \<boolean>\ (1..1)}
\\
\ruledef{T-Equals}{\Gamma |- e_1 : T_1\ C_1 \\ \Gamma |- e_2 : T_2\ C_2 \\ \lv{\comparable}(T_1\ C_1, T_2\ C_2)}{\Gamma |- e_1 \<eq> e_2 : \<boolean>\ (1..1)}
\\
\ruledef{T-NotEquals}{\Gamma |- e_1 : T_1\ C_1 \\ \Gamma |- e_2 : T_2\ C_2 \\ \lv{\comparable}(T_1\ C_1, T_2\ C_2)}{\Gamma |- e_1 \<neq> e_2 : \<boolean>\ (1..1)}
\\
\ruledef{T-AllEquals}{\Gamma |- e_1 : T_1\ C \\ \Gamma |- e_2 : T_2\ (1..1) \\ \comparable(T_1, T_2)}{\Gamma |- e_1 \<all> \<eq> e_2 : \<boolean>\ (1..1)}
\\
\ruledef{T-AllNotEquals}{\Gamma |- e_1 : T_1\ C \\ \Gamma |- e_2 : T_2\ (1..1) \\ \comparable(T_1, T_2)}{\Gamma |- e_1 \<all> \<neq> e_2 : \<boolean>\ (1..1)}
\\
\ruledef{T-AnyEquals}{\Gamma |- e_1 : T_1\ C \\ \Gamma |- e_2 : T_2\ (1..1) \\ \comparable(T_1, T_2)}{\Gamma |- e_1 \<any> \<eq> e_2 : \<boolean>\ (1..1)}
\\
\ruledef{T-AnyNotEquals}{\Gamma |- e_1 : T_1\ C \\ \Gamma |- e_2 : T_2\ (1..1) \\ \comparable(T_1, T_2)}{\Gamma |- e_1 \<any> \<neq> e_2 : \<boolean>\ (1..1)}
\\
\ruledef{T-PlusInt}{\Gamma |- e_1 : \<int>\ (1..1) \\ \Gamma |- e_2 : \<int>\ (1..1)}{\Gamma |- e_1 \<+> e_2 : \<int>\ (1..1)}
\\
\ruledef{T-PlusNumber}{\Gamma |- e_1 : \<number>\ (1..1) \\ \Gamma |- e_2 : \<number>\ (1..1)}{\Gamma |- e_1 \<+> e_2 : \<number>\ (1..1)}
\\
\ruledef{T-MultInt}{\Gamma |- e_1 : \<int>\ (1..1) \\ \Gamma |- e_2 : \<int>\ (1..1)}{\Gamma |- e_1 \<*> e_2 : \<int>\ (1..1)}
\\
\ruledef{T-MultNumber}{\Gamma |- e_1 : \<number>\ (1..1) \\ \Gamma |- e_2 : \<number>\ (1..1)}{\Gamma |- e_1 \<*> e_2 : \<number>\ (1..1)}
\\
\ruledef{T-SubtInt}{\Gamma |- e_1 : \<int>\ (1..1) \\ \Gamma |- e_2 : \<int>\ (1..1)}{\Gamma |- e_1 \<-> e_2 : \<int>\ (1..1)}
\\
\ruledef{T-SubtNumber}{\Gamma |- e_1 : \<number>\ (1..1) \\ \Gamma |- e_2 : \<number>\ (1..1)}{\Gamma |- e_1 \<-> e_2 : \<number>\ (1..1)}
\\
\ruledef{T-Division}{\Gamma |- e_1 : \<number>\ (1..1) \\ \Gamma |- e_2 : \<number>\ (1..1)}{\Gamma |- e_1 \</> e_2 : \<number>\ (1..1)}
\\
\ruledef{T-Count}{\Gamma |- e : T\ C}{\Gamma |- e \<count> : \<int>\ (1..1)}
\\
\ruledef{T-Project}{\Gamma |- e : D\ (l..u) \\ \attrs(D) = a_1\ T_1\ (l_1..u_1), \dots, a_n\ T_n\ (l_n..u_n)}{\Gamma |- e \<proj> a_k : T_k\ (l*l_k..u*u_k)}
\\
\ruledef{T-If}{\Gamma |- e_1 : \<boolean>\ (1..1) \\ \Gamma |- e_2 : T\ C \\ \Gamma |- e_3 : T\ C}{\Gamma |- \<if> e_1 \<then> e_2 \<else> e_3 : T\ C}
\\
\ruledef{T-Func}{\forall i \in 1..n: \Gamma |- e_i : T_i\ C_i \\ \inp(F) = a_1\ T_1\ C_1, \dots, a_n\ T_n\ C_n \\ \outp(F) = a\ T\ C}{\Gamma |- F(e_1, \dots, e_n) : T\ C}
\\
\ruledef{T-Construct}{\forall i \in 1..n: \Gamma |- e_i : T_i\ C_i \\ \attrs(D) = a_1\ T_1\ C_1, \dots, a_n\ T_n\ C_n}{\Gamma |- D\set{a_1 = e_1, \dots, a_n = e_n} : D\ (1..1)}
\\
\ruledef{T-Var}{x : T\ C \in \Gamma}{\Gamma |- x : T\ C}
\\
\axiomdef{T-True}{\Gamma |- \<True> : \<boolean>\ (1..1)}
\\
\axiomdef{T-False}{\Gamma |- \<False> : \<boolean>\ (1..1)}
\\
\axiomdef{T-Int}{\Gamma |- i : \<int>\ (1..1)}
\\
\axiomdef{T-Number}{\Gamma |- r : \<number>\ (1..1)}
\\
\ruledef{T-List}{\forall i \in 1..n: \Gamma |- e_i : T\ (l_i..u_i)}{\Gamma |- [e_1, \dots, e_n] : T\ (\sum\nolimits_{i \in 1..n} l_i \;.. \sum\nolimits_{i \in 1..n} u_i)}
\\
\ruledef{T-OnlyExists}{\Gamma |- e : D\ (1..1) \\ \attrs(D) = a_1\ T_1\ C_1, \dots, a_n\ T_n\ C_n}{\Gamma |- e \<proj> a_k \<only> \<exists> : \<boolean>\ (1..1)}
\\
\ruledef{T-OnlyElement}{\Gamma |- e : T\ C}{\Gamma |- e \<only-element> : T\ (0..1)}
\\
\ruledef{T-Sub}{\Gamma |- e : S \\ S <: T}{\Gamma |- e : T}
\end{align*}

Typing function declarations $F \OK$.
\begin{align*}
\ruledef{}{\inp(F) = a_1\ T_1\ C_1, \dots, a_n\ T_n\ C_n \\ \outp(F) = a\ T\ C \\ a_1 : T_1\ C_1, \dots, a_n : T_n\ C_n |- \op(F) : T\ C}{F \OK}
\end{align*}
\end{defbox}


Note: for equality, there are two sensible choices as premises. Either $\Gamma |- e_1 : T\ C_1$ and $\Gamma |- e_2 : T\ C_2$ or $\Gamma |- e_1 : T\ C$ and $\Gamma |- e_2 : T\ C$. The second possibility eliminates equality checks that are always false because the operands can never have the same length.

\subsection{Algorithmic Typing}

These typing rules should be consistent with the declarative version, but they are defined in a way that is more straightforward to implement, because every rule is syntax-directed.

\begin{defbox}
Introduce a new type $\<nothing>$ (i.e. the bottom type).

Join of basic types $\join(T_1, T_2)$.
\begin{align*}
\axiomdef{}{\join(T, T) = T}
\\
\axiomdef{}{\join(T_1, T_2) = \join(T_2, T_1)}
\\
\axiomdef{}{\join(\<int>, \<number>) = \<number>}
\\
\ruledef{}{E \in \ancestors(D)}{\join(D, E) = E}
\\
\ruledef{}{E \notin \ancestors(D) \\ \ET(E) = \<type> E \<extends> E': \dots \\ T = \join(D, E')}{\join(D, E) = T}
\\
\axiomdef{}{\join(\<nothing>, T) = T}
\end{align*}

Extension of $\join$ to $n \in \mathbb{N}$ basic types.
\begin{align*}
\axiomdef{}{\join() = \<nothing>}
\\
\axiomdef{}{\join(T) = T}
\\
\ruledef{}{n \geq 3 \\ T' = \join(T_2, \dots, T_n) \\ T = \join(T_1, T')}{\join(T_1, \dots, T_n) = T}
\end{align*}

Join for types $\lv{\join}(T_1\ C_1, T_2\ C_2)$.
\begin{align*}
\ruledef{}{T = \join(T_1, T_2) \\ C = \union(C_1, C_2)}{\lv{\join}(T_1\ C_1, T_2\ C_2) = T\ C}
\end{align*}

\subimport{}{typing-algorithmic-def}
\end{defbox}


\section{Code Generation}

\newbox\mybox
\setbox\mybox\hbox{{\lstinline|List<? |$\<extends> \stype_\java|[T|]$\lstinline|>|}}

Java representation of list types $\lv{\stype}_\java|[T\ C|]$.
\begin{align*}
\javatypedef{J-Optional}{T\ (0..1)}{\stype_\java|[T|]} \\
\javatypedef{J-Singular}{T\ (1..1)}{\begin{cases}
\lstinline|boolean|, &\text{if }T = \<boolean> \\
\lstinline|int|, &\text{if }T = \<int> \\
\stype_\java|[T|], &\text{otherwise}
\end{cases}} \\
\javatypedef{J-Plural}{T\ (k..l)}{\usebox{\mybox}, \qquad \text{if }l > 1} \\
\javatypedef{J-Empty}{T\ (0..0)}{\stype_\java|[T|]}
\end{align*}

Java reference type of each item type $\stype_\java|[T|]$.
\begin{align*}
\stype_\java|[\<boolean>|] &= \lstinline|Boolean| \\
\stype_\java|[\<int>|] &= \lstinline|Integer| \\
\stype_\java|[\<number>|] &= \lstinline|NougaNumber| \\
\stype_\java|[\<nothing>|] &= \lstinline|Void| \\
\stype_\java|[D|] &= D
\end{align*}


\end{document}
