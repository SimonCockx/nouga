\usepackage{import}

%%%%%%% AUXILIARY FUNCTIONS %%%%%%%%
\usepackage{amsmath}
\DeclareMathOperator{\ET}{ET} % entity table
\DeclareMathOperator{\FT}{FT} % function table
\DeclareMathOperator{\attrs}{attrs} % attributes
\DeclareMathOperator{\allattrs}{allattrs} % attributes, including from its ancestors
\DeclareMathOperator{\inp}{inputs} % inputs
\DeclareMathOperator{\outp}{output} % output
\DeclareMathOperator{\op}{op} % operation
\DeclareMathOperator{\fexpr}{expr} % function expression
\DeclareMathOperator{\pexpr}{expr*} % path segment expression
\DeclareMathOperator{\ancestors}{ancestors} % set of ancestors
\DeclareMathOperator{\comparable}{comparable} % check whether two basic types are comparable
\DeclareMathOperator{\overlap}{overlap} % check whether cardinalities overlap
% \DeclareMathOperator{\subcardinality}{subcardinality} % check whether a cardinality range is enclosed by another cardinality range
\newcommand{\subcard}{\subseteq}
\DeclareMathOperator{\union}{union} % union of cardinalities
\DeclareMathOperator{\join}{join} % join of two types
\DeclareMathOperator{\maybeempty}{maybeempty} % all attributes may be empty

\newcommand{\lv}[1]{#1^{*}} % list version of an operator


%%%%%%%% CODE LISTINGS %%%%%%%%%%
\usepackage[final]{listings}

\usepackage[pdftex,dvipsnames]{xcolor}
\definecolor{codegreen}{rgb}{0,0.6,0}
\definecolor{codegray}{rgb}{0.5,0.5,0.5}
\definecolor{codepurple}{rgb}{0.58,0,0.82}
\definecolor{backcolour}{rgb}{0.95,0.95,0.92}
\lstdefinestyle{mystyle}{
    backgroundcolor=\color{backcolour},   
    commentstyle=\color{codegreen},
    keywordstyle=\color{magenta},
    numberstyle=\tiny\color{codegray},
    stringstyle=\color{codepurple},
    basicstyle=\ttfamily\small,
    breakatwhitespace=false,         
    breaklines=true,                 
    captionpos=b,     
    keepspaces=true,                 
    numbers=left,                    
    numbersep=5pt,                  
    showspaces=false,                
    showstringspaces=false,
    showtabs=false,                  
    tabsize=2
}

% Add background color to inline listings
% See https://tex.stackexchange.com/a/357339/149123
% and https://tex.stackexchange.com/a/408880/149123
%\usepackage{xpatch}
%\usepackage{realboxes}
%\makeatletter
%\xpretocmd\lstinline
%  {%
%   \bgroup\fboxsep=1.5pt
%   \Colorbox{backcolour}\bgroup\kern-\fboxsep\vphantom{\ttfamily\char`\\y}%
%   \appto\lst@DeInit{\kern-\fboxsep\egroup\egroup}%
%  }{}{}
%\makeatother

% Patch for lstinline inside math mode (see https://tex.stackexchange.com/a/127018/149123)
\usepackage{letltxmacro}
\newcommand*{\SavedLstInline}{}
\LetLtxMacro\SavedLstInline\lstinline
\DeclareRobustCommand*{\lstinline}{%
  \ifmmode
    \let\SavedBGroup\bgroup
    \def\bgroup{%
      \let\bgroup\SavedBGroup
      \hbox\bgroup
    }%
  \fi
  \SavedLstInline
}

\definecolor{eminence}{RGB}{108,48,130}
\lstdefinelanguage{Rosetta}{%
alsoletter={-},
keywords={-},
otherkeywords={% Operators
+, *, /, ->, =, <>
},
morekeywords=[2]{boolean,int,number}, % types go here
morekeywords=[3]{%
type,extends,func,enum,inputs,output,assign-output,not,or,and,only,single,multiple,exists,is,absent,contains,%
disjoint,all,any,count,only-element,if,then,else,True,False,empty,%
any,boolean,string,int,number,nothing,date,dateTime,zonedDateTime,time,%
true,false % keywords go here
},%
basicstyle={\sffamily},
keywordstyle=[2]{\itshape}, % style for types
keywordstyle=[3]{\ttfamily\color{eminence}}, %style for keywords
keepspaces,
mathescape % optional
}[keywords,comments,strings]%

\lstset{style=mystyle, language=Rosetta}

%%%%%%%%% SYNTAX DEFINITIONS %%%%%%%%%%%
\usepackage[nounderscore]{syntax}
\setlength{\grammarindent}{4.5em}

\renewcommand{\grammarlabel}[2]{\synt{#1}\hfill #2}

\renewcommand{\syntleft}{$\langle$\normalfont\scshape}
\renewcommand{\syntright}{$\rangle$}

\renewcommand{\litleft}{\bgroup\normalfont\ttfamily\frenchspacing}
\renewcommand{\litright}{\egroup}

\renewcommand{\ulitleft}{\normalfont\itshape}
\renewcommand{\ulitright}{}

\newcommand{\explain}[1]{\hfill\textit{#1}\\}
\newcommand{\ind}{\phantom{.}\hspace{1em}}

%%%%%%%%% MATH LIGATURES %%%%%%%%%%
\usepackage[ligature,reserved,shorthand]{semantic}
\usepackage{stmaryrd}
\mathlig{|[}{\left\llbracket}
\mathlig{|]}{\right\rrbracket}
\reservestyle[\lit]{\rosetta}{\mathinner}
\rosetta{type,extends,func,enum,inputs,output,assign-output,not,or,and,only,single,multiple,exists,is,absent,contains,%
disjoint,all,any,count,only-element,if,then,else,True,False,empty,%
any,boolean,string,int,number,nothing,date,dateTime,zonedDateTime,time,%
true,false,eq[=],neq[<>],+,-,*,/,proj[->]}

\newcommand{\emptylist}{[\,]}

%%%%%%%% INFERENCE RULES %%%%%%%%%
\usepackage{mathpartir}
\let\oldrule\rule
\renewcommand{\rule}[1]{\textnormal{\textsc{#1}}}

\newcommand{\derivation}[1]{\mathcal{#1}}
\newcommand{\OK}{\text{\;OK}}

\usepackage{xifthen}
\newcommand{\ruledef}[3]{%
\inferrule*[vcenter,narrower=0.8]{#2}{#3}\ifthenelse{\isempty{#1}}%
    {}
    {& \quad \textsc{\small #1}}
}
%\newcommand{\axiomdef}[2]{#2 & \quad \textsc{\small #1}}
\newcommand{\axiomdef}[2]{\ruledef{#1}{ }{#2}}

\newcommand{\emphrule}[1]{{\color{red}#1}}

\newcommand{\evaldef}[3]{\sexpr|[#2|]S &= #3 && \textsc{\small #1}}
% Definition for \letin; see https://tex.stackexchange.com/a/627476/149123
\ExplSyntaxOn
\NewDocumentCommand{\letin}{mm}
 {
  \internal_letin:nn { #1 } { #2 }
 }
\seq_new:N \l__internal_letin_assign_seq
\cs_new_protected:Nn \internal_letin:nn
 {
  \seq_set_split:Nnn \l__internal_letin_assign_seq { & } { #1 }
  \openup-\jot
  \begin{aligned}[t]
  &\textbf{let~}
  \begin{aligned}[t]
  &\seq_use:Nn \l__internal_letin_assign_seq {, \\ &} \\
  \end{aligned}\\
  &\textbf{in~} #2
  \end{aligned}
 }
\ExplSyntaxOff

%%%%%%%%% SEMANTIC DEFINITION %%%%%%%%%%%
\usepackage{amssymb}
\let\oldvec\vec
\renewcommand{\vec}[1]{\boldsymbol{\mathbf{#1}}} % vector
\DeclareMathOperator{\dom}{Dom}

\newcommand{\stype}{\mathcal{T}}
\newcommand{\sexpr}{\evalsymbol}
\newcommand{\sentity}{\mathcal{D}}
\newcommand{\sprog}{\mathcal{P}}

\newcommand{\set}[1]{\left\{\,#1\,\right\}}
\newcommand{\sboolean}{\mathbb{B}}
\newcommand{\strue}{\mathit{true}}
\newcommand{\sfalse}{\mathit{false}}
\newcommand{\sint}{\mathbb{Z}}
\newcommand{\snumber}{\mathbb{R}}
\newcommand{\sprim}{\mathbb{P}}
\newcommand{\sent}{\mathbb{E}}
\newcommand{\sdom}{\mathbb{D}}

\DeclareMathOperator{\sflatten}{flatten}
\DeclareMathOperator{\scount}{count}
\DeclareMathOperator{\seq}{equals}
\newcommand{\splus}[1]{\mathbin{\mathrm{plus}}_{#1}}
\newcommand{\ssubt}[1]{\mathbin{\mathrm{subtract}}_{#1}}
\newcommand{\smult}[1]{\mathbin{\mathrm{multiply}}_{#1}}
\newcommand{\sdiv}{\mathbin{\mathrm{divide}}}
\DeclareMathOperator{\sproject}{project}


%%%%%%%%%% DEFINITION BOXES %%%%%%%%%
\usepackage[many]{tcolorbox}
\usepackage{multicol}
% Enable single column multicols
\let\multicolmulticols\multicols
\let\endmulticolmulticols\endmulticols
\RenewDocumentEnvironment{multicols}{mO{}}
 {%
  \ifnum#1=1
    #2%
  \else % More than 1 column
    \multicolmulticols{#1}[#2]
  \fi
 }
 {%
  \ifnum#1=1
  \else % More than 1 column
    \endmulticolmulticols
  \fi
 }

\newcommand{\defboxisbreakable}{true}
\AddToHook{env/spec/begin}{\renewcommand{\defboxisbreakable}{false}}
\AddToHook{env/spec/end}{\renewcommand{\defboxisbreakable}{true}}

\usepackage{xargs}
\setlength{\columnseprule}{1pt}
\newenvironmentx{defbox}[2][1=1, 2=]{%
\begin{tcolorbox}[
enhanced,
breakable=\defboxisbreakable,
sharp corners=all,
colback=black!4!white,
toprule=1.5pt,bottomrule=1.5pt,
leftrule=0pt,rightrule=0pt,
left=1pt,right=1pt,
width=0.85\paperwidth,
center,
#2
]
\begin{multicols}{#1}
\begingroup
\allowdisplaybreaks
\addtolength{\jot}{0.5em}
}
{%
\endgroup
\end{multicols}
\end{tcolorbox} 
}

\usepackage{float,caption}
\floatstyle{plaintop}
\newfloat{spec}{tbp}{specs}
\floatname{spec}{Specification}
\captionsetup[spec]{labelsep=colon}
\ifundef\chapter{}{\numberwithin{spec}{chapter}}


\usepackage{afterpage, environ}
\long\def\breakablespecbody#1[#2][#3]{%
%\afterpage%
}
\NewEnviron{breakablespec}[2]{
\expandafter\breakablespecbody\expandafter{\BODY}[#1][#2]%
}



%%%%%%%%%% CODE GENERATION %%%%%%%%%%
\newcommand{\java}{\mathcal{J}}
\DeclareMathOperator{\javaid}{javaId}
\DeclareMathOperator{\freeid}{freeId}

\newcommand{\templ}[1]{\text{\guillemotleft #1 \guillemotright}}

\lstnewenvironment{jsnippet}[1][]{%
  \endgraf\noindent\ignorespaces\lstset{
	  language=java,
	  numbers=none,
	  mathescape,
	  otherkeywords={},
	  keywordstyle={\color{eminence}},
	  #1
	}\minipage{\linewidth}
}
{%
  \endminipage\endgraf
}
\lstnewenvironment{javagen}[3][]{%
  \endgraf\noindent$#2_\java|[#3|] =$
  \endgraf\noindent\ignorespaces\lstset{
	  language=java,
	  numbers=none,
	  mathescape,
	  #1
	}\minipage{\linewidth}
}
{%
  \endminipage\endgraf
}

\floatstyle{plaintop}
\newfloat{lstfloat}{tbp}{lop}
\floatname{lstfloat}{Listing}
\captionsetup[lstfloat]{labelsep=colon}
\ifundef\chapter{}{\numberwithin{lstfloat}{chapter}}

\newcommand{\javatypedef}[3]{\lv{\stype}_\java|[#2|] &= #3 && \rule{\small #1}}

\DeclareMathOperator{\generateoutput}{genoutput}
\DeclareMathOperator{\generateexpression}{genexpr}
\newcommand{\javaexprdef}[3]{\seval_\java|[#2|] &= #3 && \rule{\small #1}}
