%%%%%%% AUXILIARY FUNCTIONS %%%%%%%%
\usepackage{amsmath}
\DeclareMathOperator{\ET}{ET} % entity table
\DeclareMathOperator{\FT}{FT} % function table
\DeclareMathOperator{\attrs}{attrs} % attributes
\DeclareMathOperator{\allattrs}{allattrs} % attributes, including from its ancestors
\DeclareMathOperator{\inp}{inputs} % inputs
\DeclareMathOperator{\outp}{output} % output
\DeclareMathOperator{\op}{op} % operation
\DeclareMathOperator{\fexpr}{expr} % function expression
\DeclareMathOperator{\pexpr}{expr*} % path segment expression
\DeclareMathOperator{\supertypes}{supertypes} % set of supertypes
\DeclareMathOperator{\comparable}{comparable} % check whether two basic types are comparable
\DeclareMathOperator{\overlap}{overlap} % check whether cardinalities overlap
\DeclareMathOperator{\subcardinality}{subcardinality} % check whether a cardinality range is enclosed by another cardinality range
\DeclareMathOperator{\union}{union} % union of cardinalities
\DeclareMathOperator{\join}{join} % join of two types

\newcommand{\lv}[1]{#1^{*}} % list version of an operator


%%%%%%%% CODE LISTINGS %%%%%%%%%%
\usepackage[final]{listings}

\usepackage[pdftex,dvipsnames]{xcolor}
\definecolor{codegreen}{rgb}{0,0.6,0}
\definecolor{codegray}{rgb}{0.5,0.5,0.5}
\definecolor{codepurple}{rgb}{0.58,0,0.82}
\definecolor{backcolour}{rgb}{0.95,0.95,0.92}
\lstdefinestyle{mystyle}{
    backgroundcolor=\color{backcolour},   
    commentstyle=\color{codegreen},
    keywordstyle=\color{magenta},
    numberstyle=\tiny\color{codegray},
    stringstyle=\color{codepurple},
    basicstyle=\ttfamily\small,
    breakatwhitespace=false,         
    breaklines=true,                 
    captionpos=b,     
    keepspaces=true,                 
    numbers=left,                    
    numbersep=5pt,                  
    showspaces=false,                
    showstringspaces=false,
    showtabs=false,                  
    tabsize=2
}

% Add background color to inline listings
% See https://tex.stackexchange.com/a/357339/149123
% and https://tex.stackexchange.com/a/408880/149123
%\usepackage{xpatch}
%\usepackage{realboxes}
%\makeatletter
%\xpretocmd\lstinline
%  {%
%   \bgroup\fboxsep=1.5pt
%   \Colorbox{backcolour}\bgroup\kern-\fboxsep\vphantom{\ttfamily\char`\\y}%
%   \appto\lst@DeInit{\kern-\fboxsep\egroup\egroup}%
%  }{}{}
%\makeatother

% Patch for lstinline inside math mode (see https://tex.stackexchange.com/a/127018/149123)
\usepackage{letltxmacro}
\newcommand*{\SavedLstInline}{}
\LetLtxMacro\SavedLstInline\lstinline
\DeclareRobustCommand*{\lstinline}{%
  \ifmmode
    \let\SavedBGroup\bgroup
    \def\bgroup{%
      \let\bgroup\SavedBGroup
      \hbox\bgroup
    }%
  \fi
  \SavedLstInline
}

\lstset{style=mystyle}
\definecolor{eminence}{RGB}{108,48,130}
\lstdefinelanguage{Rosetta}{%
extendedchars,
keywords={},
otherkeywords={% Operators
+, -, *, /, ->, =, <>
},
morekeywords=[2]{boolean,int,number}, % types go here
morekeywords=[3]{%
type,extends,func,enum,inputs,output,assign,output,not,or,and,only,single,multiple,exists,is,absent,contains,%
disjoint,all,any,count,only-element,if,then,else,True,False,empty,%
any,boolean,string,int,number,nothing,date,dateTime,zonedDateTime,time % keywords go here
},%
literate={~}{-}1,
basicstyle={\sffamily},
keywordstyle=[2]{\itshape}, % style for types
keywordstyle=[3]{\ttfamily\color{eminence}}, %style for keywords
keepspaces,
mathescape % optional
}[keywords,comments,strings]%

%%%%%%%%% SYNTAX DEFINITIONS %%%%%%%%%%%
\usepackage[nounderscore]{syntax}
\setlength{\grammarindent}{4.5em}

\renewcommand{\grammarlabel}[2]{\synt{#1}\hfill #2}

\renewcommand{\syntleft}{$\langle$\normalfont\scshape}
\renewcommand{\syntright}{$\rangle$}

\renewcommand{\litleft}{\bgroup\normalfont\ttfamily\frenchspacing}
\renewcommand{\litright}{\egroup}

\renewcommand{\ulitleft}{\normalfont\itshape}
\renewcommand{\ulitright}{}

\newcommand{\explain}[1]{\hfill\textit{#1}\\}
\newcommand{\ind}{\phantom{.}\hspace{1em}}

%%%%%%%%% MATH LIGATURES %%%%%%%%%%
\usepackage[ligature,reserved,shorthand]{semantic}
\usepackage{stmaryrd}
\mathlig{|[}{\left\llbracket}
\mathlig{|]}{\right\rrbracket}
\reservestyle[\lit]{\rosetta}{\mathinner}
\rosetta{type,extends,func,enum,inputs,output,assign-output,not,or,and,only,single,multiple,exists,is,absent,contains,%
disjoint,all,any,count,only-element,if,then,else,True,False,empty,%
any,boolean,string,int,number,nothing,date,dateTime,zonedDateTime,time,eq[=],%
neq[<>],+,-,*,/,proj[->]}

%%%%%%%% INFERENCE RULES %%%%%%%%%
\usepackage{mathpartir}
\let\oldrule\rule
\renewcommand{\rule}[1]{\textnormal{\textsc{#1}}}

\newcommand{\derivation}[1]{\mathcal{#1}}
\newcommand{\OK}{\mathinner{\text{OK}}}

\newcommand{\ruledef}[3]{\inferrule*[narrower=0.7]{#2}{#3} && \text{\small{\textsc{#1}}}}
%\newcommand{\axiomdef}[2]{\inferrule*[Right=#1]{ }{#2}}
\newcommand{\axiomdef}[2]{#2 \ && \text{\small{\textsc{#1}}}}

%%%%%%%%% SEMANTIC DEFINITION %%%%%%%%%%%
\let\oldvec\vec
\renewcommand{\vec}[1]{\boldsymbol{\mathbf{#1}}} % vector
\DeclareMathOperator{\dom}{Dom}

\newcommand{\set}[1]{\left\{\,#1\,\right\}}
\newcommand{\sboolean}{\mathbb{B}}
\newcommand{\sint}{\mathbb{Z}}
\newcommand{\snumber}{\mathbb{R}}
\newcommand{\sdom}{\mathbb{D}}
\newcommand{\ssingledom}{\mathbb{D}_1}

\newcommand{\sor}[2]{#1\,or\,#2}
\newcommand{\shor}[2]{#1\,\widehat{or}\,#2}
\newcommand{\sand}[2]{#1\,and\,#2}
\newcommand{\shand}[2]{#1\,\widehat{and}\,#2}
\newcommand{\snot}[1]{not\left(#1\right)}
\newcommand{\shnot}[1]{\widehat{not}\left(#1\right)}

\newcommand{\schoice}[3]{(#1) \rightarrow #2\,[]\,#3}
\newcommand{\shchoice}[3]{(#1) \widehat{\rightarrow} #2\,[]\,#3}
\newcommand{\scount}[1]{count\left(#1\right)}
\newcommand{\shcount}[1]{\widehat{count}\left(#1\right)}
\newcommand{\sflatten}[3]{flatten_{#1, #2}\left(#3\right)}
\newcommand{\shflatten}[3]{\widehat{flatten}_{#1, #2}\left(#3\right)}
\newcommand{\sproject}[3]{project_{#1, #2}\left(#3\right)}
\newcommand{\shproject}[3]{\widehat{project}_{#1, #2}\left(#3\right)}
\newcommand{\scontains}[2]{contains\left(#1, #2\right)}
\newcommand{\shcontains}[2]{\widehat{contains}\left(#1, #2\right)}
\newcommand{\sdisjoint}[2]{disjoint\left(#1, #2\right)}
\newcommand{\shdisjoint}[2]{\widehat{disjoint}\left(#1, #2\right)}
\newcommand{\sonlyexists}[3]{onlyexists_{#1, #2}\left(#3\right)}
\newcommand{\shonlyexists}[3]{\widehat{onlyexists}_{#1, #2}\left(#3\right)}
\newcommand{\sonlyelement}[1]{onlyelement\left(#1\right)}
\newcommand{\shonlyelement}[1]{\widehat{onlyelement}\left(#1\right)}

\newcommand{\seq}[2]{#1\,eq\,#2}
\newcommand{\sheq}[2]{#1\,\widehat{eq}\,#2}
\newcommand{\sneq}[2]{#1\,neq\,#2}
\newcommand{\shneq}[2]{#1\,\widehat{neq}\,#2}
\newcommand{\sanyeq}[2]{#1\,anyeq\,#2}
\newcommand{\shanyeq}[2]{#1\,\widehat{anyeq}\,#2}
\newcommand{\sanyneq}[2]{#1\,anyneq\,#2}
\newcommand{\shanyneq}[2]{#1\,\widehat{anyneq}\,#2}
\newcommand{\salleq}[2]{#1\,alleq\,#2}
\newcommand{\shalleq}[2]{#1\,\widehat{alleq}\,#2}
\newcommand{\sallneq}[2]{#1\,allneq\,#2}
\newcommand{\shallneq}[2]{#1\,\widehat{allneq}\,#2}

\newcommand{\splus}[3]{#2\,plus_{#1}\,#3}
\newcommand{\shplus}[3]{#2\,\widehat{plus}_{#1}\,#3}
\newcommand{\ssubt}[3]{#2\,subtract_{#1}\,#3}
\newcommand{\shsubt}[3]{#2\,\widehat{subtract}_{#1}\,#3}
\newcommand{\smult}[3]{#2\,mult_{#1}\,#3}
\newcommand{\shmult}[3]{#2\,\widehat{mult}_{#1}\,#3}
\newcommand{\sdiv}[2]{#1\,div\,#2}
\newcommand{\shdiv}[2]{#1\,\widehat{div}\,#2}


%%%%%%%%%% DEFINITION BOXES %%%%%%%%%
\usepackage[many]{tcolorbox}
\usepackage{multicol}
% Enable single column multicols
\let\multicolmulticols\multicols
\let\endmulticolmulticols\endmulticols
\RenewDocumentEnvironment{multicols}{mO{}}
 {%
  \ifnum#1=1
    #2%
  \else % More than 1 column
    \multicolmulticols{#1}[#2]
  \fi
 }
 {%
  \ifnum#1=1
  \else % More than 1 column
    \endmulticolmulticols
  \fi
 }

\newcommand{\defboxisbreakable}{true}
\AddToHook{env/spec/begin}{\renewcommand{\defboxisbreakable}{false}}
\AddToHook{env/spec/end}{\renewcommand{\defboxisbreakable}{true}}

\usepackage{xargs}
\setlength{\columnseprule}{1pt}
\newenvironmentx{defbox}[2][1=1, 2=]{%
\begin{tcolorbox}[
enhanced,
breakable=\defboxisbreakable,
sharp corners=all,
colback=black!4!white,
toprule=1.5pt,bottomrule=1.5pt,
leftrule=0pt,rightrule=0pt,
left=1pt,right=1pt,
width=0.85\paperwidth,
center,
#2
]
\begin{multicols}{#1}
\begingroup
\allowdisplaybreaks
\addtolength{\jot}{0.5em}
}
{%
\endgroup
\end{multicols}
\end{tcolorbox} 
}

\usepackage{float,caption}
\floatstyle{plaintop}
\newfloat{spec}{tbp}{specs}
\floatname{spec}{Specification}
\captionsetup[spec]{labelsep=colon}
\numberwithin{spec}{chapter}


%%%%%%%%%% CODE GENERATION %%%%%%%%%%
\newcommand{\java}{\mathcal{J}}
\DeclareMathOperator{\javaid}{javaId}
\DeclareMathOperator{\freeid}{freeId}
